\section{Einleitung} \thispagestyle{nomarkstyle}
Die Ausbreitung des elektronischen Handels ging mit der Erfolgsgeschichte und Kommerzialisierung des Internets einher und wächst seitdem immer weiter.
Für viele Endkunden ist der Einkauf von Waren in Online-Shops alltäglich geworden und bietet im Vergleich zum konventionellen Handel viele Vorteile.
Große Versandhäuser wie Amazon bieten weltweit ihre Dienste an und reagieren auf die sich ändernde Nachfrage ihrer Kunden.
Wo früher Preisvergleiche und Informationsbeschaffung zu bestimmten Produkten einen erheblichen Zeitaufwand bedeuteten und nicht an einem Ort zu finden waren, bieten heute Suchmaschinen ihre Dienste an und ermöglichen Nutzern bequem von zu Hause aus diese Informationen abzurufen.
Davon profitieren wiederum die Online-Händler, die dadurch Kunden gewinnen können.
Einzelhändler und Kaufhäuser haben diese Entwicklung schmerzlich erfahren, da sich der Preisdruck erhöht hat und der E-Commerce zudem weniger Nebenkosten benötigt.
Viele traditionelle Händler haben aber reagiert und ihr Geschäft durch eigene Online-Shops ergänzt. \cite{Riehm}

Die Gestaltung und technischen Umsetzungsmöglichkeiten von Webshops haben sich mit der Zeit ebenso weiterentwickelt.
Heute gibt es eine Vielzahl an Technologien und Herangehensweisen, um einen Online-Handel aufzusetzen und zu betreiben.

Diese Studienarbeit beschäftigt sich mit der Konzeption und Umsetzung eines solchen Webshops.

\subsection{Aufgabenstellung}
Die Erstellung eines Webshops mit der benötigten Infrastruktur ist die Aufgabenstellung der vorliegenden Arbeit.
Dabei soll es einen Bereich für Benutzer (oder Kunden) für den Einkauf im Shop geben, sowie einen Bereich für Administratoren zur Verwaltung des Bestands und der Erfassung neuer Artikel.

Details zur Funktionalität und inhaltlichen Gestaltung finden sich in \cref{specification}.
\subsection{Projektziele}
Neben der Umsetzung der Anforderungen und Funktionalitäten für den Webshop gibt es weitere Ziele für das Projekt.
Dazu gehört der Wunsch, neue Technologien zu evaluieren, einzusetzen und sich anzueignen, was auch in einem professionellen Umfeld immer wieder nötig ist, um dem schnellen Wandel und der Entwicklung der technischen Möglichkeiten gerecht zu werden.
Außerdem soll für die gemeinsame Projektarbeit auf eine strukturierte Arbeitsweise geachtet, sowie best practices herausgearbeitet werden, um einen möglichst reibungslosen und effektiven Ablauf des Projekts zu gewährleisten.
