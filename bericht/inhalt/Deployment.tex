\section{Deployment} \thispagestyle{nomarkstyle}
Im Verlauf der Entwicklung stellte sich intern das Bedürfnis nach einer durchgehend lauffähigen Online-Demo des Webshops heraus. Dabei sollten alle Änderungen am Projekt automatisch compiliert und in das Demo-System eingespielt werden. Um dies umsetzen zu können wurde eine Deployment-Strategie definiert, die in diesem Abschnitt näher erläutert wird.

\subsection{Versionsverwaltung mit Git}
Für die Versionsverwaltung des Projekts wird Git verwendet, eines der beliebtesten Versionsverwaltungssysteme in der Softwareentwicklung. Als zentrale Ablage des Quellcodes wurde ein Repository in der GitHub-Plattform erstellt, sodass die Teammitglieder bequem ihre Änderungen an das Repository senden und auch die neuesten Änderungen aktualisieren können\cite{Coutermarsh2014}. Neben den allgemeinen Vorteilen von Git und GitHub ist deren Einsatz auch eine Voraussetzung für die nächsten Schritte.

\subsection{Travis Continuous Integration}
Die Travis CI (Continuous Integration) Software ermöglicht es automatisch bei einem Commit auf das GitHub Repository die Anwendung zu bauen (inklusive Test-Suites falls welche konfiguriert sind). Zudem kann Travis auch so eingestellt werden, dass nach einem erfolgreichen Build der Code direkt in die gewünschte Umgebung deployed wird\cite{Coutermarsh2014}.

Die Einrichtung von Travis ist in wenigen Schritten erledigt:

\begin{enumerate}
	\item Je nachdem ob es sich um ein Open Source oder ein privates Repository handelt ist eine Anmeldung mit dem GitHub-Konto über jeweils \textit{travis-ci.org} oder \textit{travis-ci.com} nötig. In diesem Fall handelt es sich um ein Open Source Projekt.
	\item In der Profilübersicht sind alle Repositories des angemeldeten Benutzerkontos aufgelistet. Anhand eines Buttons kann Travis für das entsprechende Repository aktiviert werden. Dadurch wird ein sogenanntes \enquote{Webhook} gesetzt, welches zukünftige Commits erkennt und automatisch ein Build ausführt.
	\item Im Hauptverzeichnis des Repositories muss eine Datei mit dem Namen \texttt{.travis.yml} erstellt werden. Diese Datei teilt Travis mehrere Informationen die für den Build benötigt werden, wie die verwendete Programmiersprache, weitere Befehle die ausgeführt werden sollen und die Deployment-Konfigurationen.
\end{enumerate}

Der Inhalt der \textit{.travis.yml}-Datei für dieses Projekt ist in \cref{travis_yaml} zu finden.


\subsection{Heroku}
%TODO Free dyno, Eigenschaften, setup, Möglichkeiten
\cite{Coutermarsh2014}
\subsection{AWS Datenbank Instanz}
\cite{Gaut2016}