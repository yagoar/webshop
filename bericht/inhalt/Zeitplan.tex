\section{Planung} \thispagestyle{nomarkstyle}
Für die Bearbeitung des Projekts ist ein Zeitraum von 21 Wochen vorgesehen, von KW43/2016 bis zum Abgabetermin am 2. Juni 2017.
Die Planung der einzelnen Arbeitsschritte wird in den nächsten Abschnitten beschrieben.

\subsection{Projektphasen}
Die Planung des Projekts ist grundsätzlich in fünf Phasen unterteilt:
\begin{enumerate}
	\item Evaluierung der möglichen Technologien und Architekturentscheidung. Die Ergebnisse dieser Phase sind in \cref{web_architecture,arch_decission} zu finden.
	\item Vorbereitung des Projekts. Dazu gehören das Aufsetzen des Backends, der Datenbank und des Frontends, sowie die Verbindung aller Komponenten.
	\item Umsetzung der Benutzerschnittstelle des Shops und für den Admin-Bereich.
	\item Umsetzung der serverseitigen Schnittstellen samt der darunterliegenden Logik.
	\item Verbindung von Benutzerschnittstelle zu den verfügbaren Schnittstellen im Server.
\end{enumerate}

Hierbei ist vorgesehen, dass Phasen 3 und 4 weitestgehend parallel ablaufen. Eine visuelle Darstellung der Phasen, sowie eine feinere Aufteilung der Aufgaben in jeder Phase können im Anhang eingesehen werden. %TODO Anhang Meilensteine

\subsection{Meilensteine}
Mit der Planung jeder Phase wurden Meilensteine definiert, welche das Erreichen von bestimmten Zwischenzielen darstellen. Somit kann während des Projekts jederzeit der Fortschritt kontrolliert werden.
Im Rahmen dieses Projekts sind diese Meilensteine jedoch nur als grobe Orientierung gedacht, da Abweichungen durch externe Variablen durchaus die Planung verändern können.

Die festgelegte Meilensteine sind der \cref{tab:milestones} zu entnehmen.

\begin{table}[ht!]
\begin{tabular}{|p{0.6\textwidth}|p{0.3\textwidth}|}
	\hline 
	\textbf{Meilenstein} & \textbf{Datum} \\ 
	\hline 
	Architektur festgelegt & 02.12.2016 \\ 
	\hline 
	Projektgrundlage bereit & 13.12.2016 \\ 
	\hline 
	Grundlegendes Datenmodell definiert & 16.12.2016 \\ 
	\hline 
	Entitäts-Klassen des Datenmodells mit Datenbank-Mapping implementiert & 22.02.2017 \\ 
	\hline 
	Grundlegende Shop-Benutzerschnittstelle implementiert & 21.03.2017 \\ 
	\hline 
	Grundlegende Administrations-Benutzerschnittstelle implementiert & 04.04.2017 \\ 
	\hline 
	Grundlegende Schnittstellen für den Shop implementiert & 31.01.2017 \\ 
	\hline 
	Grundlegende Schnittstellen für den Admin-Bereich implementiert & 14.04.2017 \\ 
	\hline 
	Services im Front-End implementiert	& 21.04.2017 \\ 
	\hline 
\end{tabular} 
\caption{Projektmeilensteine} \label{tab:milestones}
\end{table}


\subsection{Aufgabenverteilung}
Die Aufgaben werden im Projekt so verteilt, dass es möglichst wenig Abhängigkeiten zwischen den Teammitgliedern gibt. Diese Trennung wird erreicht, indem die Aufgaben in Backend und Frontend Entwicklung unterteilt werden und die Teammitglieder jeweils eine Entwicklungsstelle übernehmen.

Dabei wird Jens Gerle hauptsächlich die Programmierung des Backends angehen und Yaiza Gonzalo Alt die des Frontends.
Diese Aufteilung ist jedoch nicht zwingend einzuhalten und kann je nach Bedarf oder Interesse angepasst werden.















