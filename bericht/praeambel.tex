\documentclass[a4paper,12pt]{article}

\usepackage[english,ngerman]{babel}

% Seitenlayout
\usepackage[onehalfspacing]{setspace}
\usepackage[left=2.5cm,right=2.5cm,top=3cm,bottom=3cm,headheight=15pt]{geometry}
\setlength{\parindent}{0pt}
\setlength{\parskip}{1em}
\setlength{\emergencystretch}{\textwidth}
\renewcommand{\baselinestretch}{1.5}\normalsize
\usepackage{setspace}

% Schriftart
\usepackage{upquote}
\usepackage[utf8]{inputenc}
\usepackage[T1]{fontenc}
\usepackage[scaled]{helvet}
\renewcommand*{\familydefault}{\sfdefault}

%Inhaltsverzeichnis
\setcounter{tocdepth}{2}
\usepackage{tocvsec2}
\usepackage{tocloft} 
\setlength\cftparskip{-2pt}
\setlength\cftbeforesecskip{1pt}
\setlength\cftaftertoctitleskip{2pt}

%Literaturverzeichnis
\usepackage{csquotes}
\usepackage[backend=biber,bibencoding=utf8,sorting=none]{biblatex}
\addbibresource{studienarbeit.bib}

%Abkürzungsverzeichnis
\usepackage[printonlyused]{acronym}

%Anhang
\usepackage[titletoc,page]{appendix}
\newcommand*{\Appendixautorefname}{Anhang}

%PDFs
\usepackage{pdfpages}

%Graphiken / Tabellen
\usepackage{graphicx}
\usepackage{wrapfig}
\usepackage{caption}
\captionsetup[figure]{font=small,skip=3pt}
\captionsetup[table]{font=small,skip=3pt}
\captionsetup[lstlisting]{font=small,skip=3pt}
\usepackage{array}
\renewcommand{\arraystretch}{1.5}
\usepackage{framed}
\usepackage{longtable}
\usepackage{color}

%Abschnitte
\usepackage[compact]{titlesec}
\let\subsectionautorefname\sectionautorefname
\let\subsubsectionautorefname\sectionautorefname
\titleformat{\section}
{\normalfont\sffamily\Huge\bfseries}
{\thesection}{1em}{}
\titleformat{\subsection}
{\normalfont\sffamily\Large\bfseries}
{\thesubsection}{1em}{}


%Abstract
\usepackage{abstract}

%Header und Footer
\usepackage{fancyhdr}
\pagestyle{fancy}

\fancyhf{} 
\fancyfoot[L]{ \today }
\fancyfoot[R]{\thepage}
\renewcommand{\headrulewidth}{0pt}
\renewcommand{\footrulewidth}{0pt}

\fancypagestyle{fancynopage}{
	\fancyhf{} 
	\fancyfoot[L]{\today}
	\renewcommand{\headrulewidth}{0pt}
	\renewcommand{\footrulewidth}{0pt}
}

\fancypagestyle{fullstyle}{
\fancyhf{} 
\fancyhead[L]{\rightmark}
\fancyfoot[L]{\today}
\fancyfoot[R]{\thepage}
\renewcommand{\headrulewidth}{0pt}
\renewcommand{\footrulewidth}{0pt}
}

\fancypagestyle{nomarkstyle}{
	\fancyhf{} 
	\fancyfoot[L]{\today}
	\fancyfoot[R]{\thepage}
	\renewcommand{\headrulewidth}{0pt}
	\renewcommand{\footrulewidth}{0pt}
}

\fancypagestyle{plain}{
	\thispagestyle{fancy}
}

%Listings
\usepackage{listings}

\definecolor{editorLightGray}{cmyk}{0.05, 0.05, 0.05, 0.1}
\definecolor{editorGray}{cmyk}{0.6, 0.55, 0.55, 0.2}
\definecolor{editorPurple}{cmyk}{0.5, 1, 0, 0}
\definecolor{editorWhite}{cmyk}{0, 0, 0, 0}
\definecolor{editorBlack}{cmyk}{1, 1, 1, 1}
\definecolor{editorOrange}{cmyk}{0, 0.8, 1, 0}
\definecolor{editorBlue}{cmyk}{1, 0.6, 0, 0}
\definecolor{editorPink}{cmyk}{0, 1, 0, 0}

\definecolor{javared}{rgb}{0.6,0,0} % for strings
\definecolor{javagreen}{rgb}{0.25,0.5,0.35} % comments
\definecolor{javapurple}{rgb}{0.5,0,0.35} % keywords
\definecolor{javadocblue}{rgb}{0.25,0.35,0.75} % javadoc

\lstdefinelanguage{HTML5}{
	language=html,
	tagstyle=\color{editorBlue},
	identifierstyle=\color{editorBlack},
	keywordstyle=\color{editorBlue},
	commentstyle=\color{editorGray},
	stringstyle=\color{editorPurple},
	moredelim=[s][{\color{editorOrange}}]{(}{)},
	moredelim=[s][{\color{editorOrange}}]{[}{]}
}

\lstdefinelanguage{json}{ 
	literate= 
	*{:}{{{\color{editorBlue}{:}}}}{1} 
	{,}{{{\color{editorBlue}{,}}}}{1} 
	{\{}{{{\color{editorBlue}{\{}}}}{1} 
	{\}}{{{\color{editorBlue}{\}}}}}{1} 
	{[}{{{\color{editorBlue}{[}}}}{1} 
	{]}{{{\color{editorBlue}{]}}}}{1}
} 

\lstdefinelanguage{JavaScript}{
	keywords={typeof, new, true, false, catch, function, return, null, catch, switch, var, if, in, while, do, else, case, break, let, const},
	keywordstyle=\color{editorBlue}\bfseries,
	ndkeywords={class, export, boolean, throw, implements, import, this},
	ndkeywordstyle=\color{editorBlue}\bfseries,
	identifierstyle=\color{editorBlack},
	sensitive=false,
	comment=[l]{//},
	morecomment=[s]{/*}{*/},
	commentstyle=\color{editorGray}\ttfamily,
	stringstyle=\color{editorPurple}\ttfamily,
	morestring=[b]',
	morestring=[b]",
	moredelim=[s][{\color{editorPurple}}]{@Component(\{}{\})},
	moredelim=[s][{\color{editorPurple}}]{@NgModule(\{}{\})}
}

\lstset{language=Java,
	basicstyle=\ttfamily,
	keywordstyle=\color{javapurple}\bfseries,
	stringstyle=\color{javared},
	commentstyle=\color{javagreen},
	morecomment=[s][\color{javadocblue}]{/**}{*/},
	numbers=left,
	numberstyle=\tiny\color{black},
	stepnumber=2,
	numbersep=8pt,
	tabsize=4,
	showspaces=false,
	showstringspaces=false
}

\makeatletter

% here is a macro expanding to the name of the language
% (handy if you decide to change it further down the road)
\newcommand\language@yaml{yaml}

\expandafter\expandafter\expandafter\lstdefinelanguage
\expandafter{\language@yaml}
{
	keywords={true,false,null,y,n},
	keywordstyle=\color{editorGray},
	sensitive=false,
	comment=[l]{\#},
	morecomment=[s]{/*}{*/},
	commentstyle=\color{editorPurple}\ttfamily,
	stringstyle=\color{editorPurple}\ttfamily,
	moredelim=[l][\color{editorPink}]{*},
	moredelim=**[il][\color{editorOrange}{:}\color{editorBlack}]{:},   % switch to value style at :
	morestring=[b]',
	morestring=[b]",
	literate =    {---}{{\ProcessThreeDashes}}3
	{>}{{\textcolor{red}\textgreater}}1     
	{|}{{\textcolor{red}\textbar}}1 
	{\ -\ }{{\mdseries\ -\ }}3,
}

% switch to key style at EOL
\lst@AddToHook{EveryLine}{\ifx\lst@language\language@yaml\color{editorBlue}\fi}
\makeatother

\newcommand\ProcessThreeDashes{\llap{\color{editorGray}\mdseries-{-}-}}

\lstset{
	basicstyle=\footnotesize\ttfamily\setstretch{1},
	numbers=left,
	numberstyle=\scriptsize,
	stepnumber=1,
	numbersep=8pt,
	frame=single,
	captionpos=b
}

\lstset{literate=%
	{Ö}{{\"O}}1
	{Ä}{{\"A}}1
	{Ü}{{\"U}}1
	{ß}{{\ss}}1
	{ü}{{\"u}}1
	{ä}{{\"a}}1
	{ö}{{\"o}}1
}

%Referenzen
\usepackage[hidelinks]{hyperref}
\usepackage[german]{cleveref}
