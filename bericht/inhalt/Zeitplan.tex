\section{Planung} \thispagestyle{nomarkstyle}
Für die Bearbeitung des Projekts sind circa 21 Wochen vorgesehen, von KW43/2016 bis KW22/2017. Offizieller Abgabetermin ist am 2. Juni 2017. Die genauere Planung der Arbeitsschritten wird in den nächsten Unterabschnitten beschrieben.

\subsection{Projektphasen}
Die Planung des Projekts ist grundsätzlich in vier Phasen unterteilt:
\begin{enumerate}
	\item Evaluierung der möglichen Technologien und Architekturentscheidung. Die Ergebnisse dieser Phase sind in \cref{web_architecture,arch_decission} zu finden.
	\item Vorbereitung des Projekts. Dazu gehören das Aufsetzen des Back-Ends, der Datenbank und des Front-Ends, sowie die Verbindung aller Komponenten.
	\item Umsetzung der  Benutzerschnittstelle sowohl des Shop selbst wie für den Admin-Bereich.
	\item Umsetzung der serverseitigen Schnittstellen samt der darunterliegenden Logik.
	\item Verbindung von Benutzerschnittstelle zu den verfügbaren Schnittstellen im Server.
\end{enumerate}

Hierbei ist es vorgesehen, dass Phasen 3 und 4 weitestgehend parallel ablaufen. Eine visuelle Darstellung der Phasen, sowie eine feinere Aufteilung der Aufgaben in jeder Phase können im Anhang eingesehen werden.

\subsection{Meilensteine}
Mit der Planung jeder Phase entstehen Meilensteine, welche das Erreichen von bestimmten Zwischenzielen darstellen. Somit kann während des Projekts jederzeit der Fortschritt kontrolliert werden. Im Rahmen dieses Projekts sind diese Meilensteine jedoch nur als grobe Orientierung gedacht, da Abweichungen durch externe Variablen durchaus die Planung verändern können.

Die festgelegte Meilensteine sind der \cref{tab:milestones} zu entnehmen.

\begin{table}[ht!]
\begin{tabular}{|p{0.6\textwidth}|p{0.3\textwidth}|}
	\hline 
	\textbf{Meilenstein} & \textbf{Datum} \\ 
	\hline 
	Architektur festgelegt & 02.12.2016 \\ 
	\hline 
	Projektgrundlage bereit & 13.12.2016 \\ 
	\hline 
	Grundlegendes Datenmodell definiert & 16.12.2016 \\ 
	\hline 
	Entitäts-Klassen des Datenmodells mit Datenbank-Mapping implementiert & 22.02.2017 \\ 
	\hline 
	Grundlegende Shop-Benutzerschnittstelle implementiert & 21.03.2017 \\ 
	\hline 
	Grundlegende Administrations-Benutzerschnittstelle implementiert & 04.04.2017 \\ 
	\hline 
	Grundlegende Schnittstellen für den Shop implementiert & 31.01.2017 \\ 
	\hline 
	Grundlegende Schnittstellen für den Admin-Bereich implementiert & 14.04.2017 \\ 
	\hline 
	Services im Front-End implementiert	& 21.04.2017 \\ 
	\hline 
\end{tabular} 
\caption{Projektmeilensteine} \label{tab:milestones}
\end{table}


\subsection{Aufgabenverteilung}
Die Aufgaben werden im Projekt so verteilt, dass es möglichst wenig Abhängigkeiten zwischen den Teammitgliedern gibt. Diese Trennung wird erreicht, indem die Aufgaben in Back-End und Front-End Entwicklung unterteilt werden und die Teammitglieder jeweils eine Entwicklungsstelle übernehmen.

Voraussichtlich wird Jens Gerle hauptsächlich die Programmierung des Back-Ends angehen und Yaiza Gonzalo Alt die des Front-Ends. Diese Aufteilung ist jedoch nicht zwingend einzuhalten und kann je nach Bedarf oder Interesse angepasst werden.















